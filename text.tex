\documentclass[10pt,a4paper,notitlepage]{article}

\usepackage[left=1.5cm,text={18cm, 25cm},top=2.5cm]{geometry}
\usepackage{color}
\usepackage[utf8]{inputenc}

\author{Michal Šrubař\\xsruba03@stud.fit.vutbr.cz}
\date{}	
\title{Zdokonalení integrace SSSD a SUDO}

\begin{document}
\maketitle

Jedná se pouze o soupis znalostí, abych si ujasnil co už vím a co jsem pochopil
špatně.

\section{Možnosti administrace unix-like operačního systému}
Na většině dnešních UNIX-like operačních systémech jsou všechny administrativní
úlohy prováděny uživatelem \textit{root}. Pouze root, a nikdo jiný, má nad
systémem plnou kontrolu. Jestliže systém spravovat jiný uživatel, pak musí znát
heslo uživatele root nebo využí nástroje sudo.

\subsection{su (Substitute User)}
Program \textbf{su} umožnuje uživateli se dočasně stát jiným uživatelem. Ve
většině případů je tím jiným uživatelem právě uživatel \textbf{root}. Su tedy
může libovolnému uživateli dát plnou kontrolu nad systémem, jestliže daný
uživatel zná heslo uživatele \textbf{root}. Su umí logovat události pomocí
syslog, pokud je s touto podporou zkompilován.

\subsection{sudo}
Sudo umožňuje uživateli vykonat jeden příkaz jako jiný uživatel. Většinou je
jiný uživatel administrátor systému root.  Bezpečnostní politiky, kterými se
sudo řídí, tj. sudo pravidla, jsou uložena v souboru \texttt{/etc/sudoers}. Na
každém systému, kde chci sudo využít musí být tento konfigurační soubor
existovat.  Tento soubor sice může být sdílen mezi více systémy, ale neexistuje
nativní způsob jak jej mezi více systémy distribuovat. Ve většině firemních
prostředí se dnes používá adresářových služeb k synchronizaci uživatelů, skupin
a dalších sdílených informací. Sudo pravidla mohou být na tomto serveru také
uložena.  Přistoupit k nim můžeme pomocí protokolu LDAP\footnote{Lightweight
	Directory Access Protocol, which is an application protocol for querying and
modifying directory services.}. Při použití adresářového serveru je možné sudo
pravidla centrálně spravovat a globálně k nim přistupovat. Sudo ovšem takto
nativně nepracuje proto je potřeba plugin. Sudo má plugin pro použítí s
openLDAP. Schéma v jakém jsou sudo pravidla na LDAP serveru uložena je pevně
specifikováno a přináší následující nevýhody: 
\begin{enumerate} 
	\item problém s	tím překrýváním lokálních uživatelů 
\end{enumerate}

Pokud chce administrátor tento způsob správy politik využít, pak musí dané
schéma dodržet i se všemi jeho nevýhodami.

\subsection{su vs. sudo}
Jedná se dva naprosto odlišné přístupy k tomu jak získat oprávnění uživatele
root.

\begin{enumerate}
	\item sudo umožňuje limitovat přístup zatímco su může uživateli dát plnou
		kontrolu nad systémem
	\item uživatel nemusí znát heslo uživatele root
	\item použití sudo může být monitorováno
	\item je možné definovat, kteří uživatelé mohou provádět jaké úlohy
\end{enumerate}



\subsection{FreeIPA}
FreeIPA používá vlastní schéma a proto musela napsat i vlastní plugin sss.


Výhody použití FreeIPA pro správu sudo pravidel - u sudo+openldap musím vytvořit
schéma a ručně jej přidat mezi ostatní schémata + přidat o tom záznam do
/etc/sldap.conf


Sudo Plugin API\footnote{http://www.sudo.ws/sudo\_plugin.man.html}
umožňuje vytvoření vlastního modulu, který bude definovat vlastní správu
politik.  To jaké pluginy bude sudo používat je možné konfigurovat pomocí
souboru \textit{/etc/sudo.conf}.





\end{document}
