\documentclass[10pt,a4paper,notitlepage]{article}

\usepackage[left=1.5cm,text={18cm, 25cm},top=2.5cm]{geometry}
\usepackage{color}
\usepackage[utf8]{inputenc}

\author{Michal Šrubař\\xsruba03@stud.fit.vutbr.cz}
\date{}	
\title{Zdokonalení integrace SSSD a SUDO}

\begin{document}
\maketitle

\section*{Možná struktura textu}
\begin{itemize}
	\item	Smysl BP.
	\item Uživatelé, root, jak dát běžným uživatelům vyšší práva
	\item Sudo, su, alternativy
	\item Proč sudo?
	\item	Proč sudo v LDAPu? (alternativy jak a kam ukládát sudo pravidla, např.
		puppet)

	\begin{itemize}
		\item Složitost konfigurace a správy (všechno v textu, není GUI, které FreeIPA má)
		\item Firma s mnoha zaměstnanci chce centrálně spravovat informace a
			politiky o svých uživatelích (chtěla by mít vše na jednom místě a né
			informace v ldap, sudo pravidla v LDAPu ale nebylo by to nijak propojeno,
			konfiguráky v puppetu, atd...)
		\item sudo pravidla nejsou nijak propojena s ostatními informacemi, které
			mohou a často jsou v LDAPu uloženy (uživatelé, skupiny, atd)
	\end{itemize}
	
	\item Problémy a nedostatky defaultního LDAP shématu, které SUDO používá
	\begin{itemize}
		\item na uživatele se odkazuje pouze přes řetězec jména, není tam žádná
			provázanost s počítačem ze kterého dany uživatel je
		\item pokud LDAP spravuje i uživatele, a je tam stejný uživatel jako na
			lokální mašině, tak tam vzniká nějaký problém s překrytím těch uživatelů
		\item mezi uživateli a sudo pravidly neexistuje vazba, pokud odstraním
			uživatele, tak mně zůstane pravidlo s neexistujícím uživatelem
	\end{itemize}
	\item Projekt FreeIPA
	\item Proč sudo ve FreeIPA (?granularita?)
	\item Proč má IPA vlastní SUDO LDAP schema a jaké výhody přináší

	\begin{itemize}
		\item umožňuje např. povolit/zakázat pravidlo
	\end{itemize}
		
	\item Jak to funguje teď

	\begin{itemize}
		\item definice pravidla
		\item jak se uloží v LDAPu
		\item co tam bude jinak oproti tomu než bych to uložil do LDAPu s legacy
			schematem
		\item konverze do legacy schématu když k tomu přistoupím přes SSSD
		\item cachování
	\end{itemize}
	\item Proč musí IPA nové schéma přeložit do legacy schematu než jej pošle
		SSSD
	\item Jaké problémy to přináší
	\item Jaké úpravy je nutné provést
	\begin{itemize}
		\item u sudo bude třeba přenést něco z /lib do /usr (z ticketu: pe\_task,
			cron)
	\end{itemize}
	\item Vyzdvihnout všechny přednosti
\end{itemize}

\section*{Hlavní problém, který moje BP řeší}
FreeIPA používá pro uložení SUDO pravidel v LDAPu svoje schéma, které přináší
mnoho vylepšení. \textbf{SSSD} ovšem toto schéma neumí zpracovat (FreeIPA
nejprve používala legacy schema pro uložení a pak navrhla nové). FreeIPA musí
nové schéma přeložit do legacy schématu, které sssd umí zpracovat. Převodem se
ovšem ztrácí mnoho informací, které potom způsobují problémy a brzí to tak
využítí celého potenciálu.

Jelikož SSSD neumí to FreeIPA schéma, tak se musí dodatečně stahovat mnoho
informací, což u rozsáhlé databáze způsobuje značné zpomalení. Úkolem není
navrhnout nové LDAP schéma pro SUDO, ale upravit SSSD daemon a SUDO tak, aby
podporovaly LDAP schéma, které používá IPA.
\\
\\
Hodně zjednodušeně bude výsledkem správný dotaz na LDAP adresář na IPA serveru,
který pro daného klienta stáhne všechno potřebné.


\section{Administrace linuxového systému}
Na většině dnešních linuxových a
Unix-like\footnote{http://en.wikipedia.org/wiki/Unix-like} operačních systémech
provádí správu systému speciální uživatel zvaný \textbf{root}. Někdy také
označován jako \textbf{superuser}. Tento uživatel má přístup ke všem souborům a
může spouštět všechny příkazy operačního
systému\footnote{http://www.linfo.org/root.html}.  Ostatní uživatele systému
toto právo nemají, až na tyto vyjímky: \begin{itemize}
	\item uživatel zná heslo uživatele root
	\item uživatel použije program, který mu dočasně poskytne práva uživatele root
\end{itemize}


\subsection{su (Substitute User)}
Tento program umožňuje libovolnému uživateli stát se dočasně jiným uživatelem.
Vě většině případů je jiný uživatel právě uživatel root. To znamená, že program
su může dát libovolnému uživateli, který zná heslo uživatele root, plnou
kontrolu nad operačním systémem.


\subsection{sudo (Super User DO)}
Sudo umožňuje uživateli provést jeden příkaz jako jiný uživatel. Bezpečností
politiky, které definují kteří uživatelé mohou spouštět jaké příkazy se nazývají 
\textbf{sudo pravidla}. Tyto pravidla jsou uloženy v souboru
\emph{/etc/sudoers}. Sudo pravidla by měla být upravována výhradně pomocí
utility \textbf{visudo}, která zároveň provádí kontrolu syntaxe souboru
\emph{/etc/sudoers}.

\subsubsection{Mezi hlavní výhody SUDO}
Jedná se dva naprosto odlišné přístupy k tomu jak získat oprávnění uživatele
root.

\begin{itemize}
	\item sudo umožňuje limitovat přístup zatímco su může uživateli dát plnou
		kontrolu nad systémem
	\item uživatel nemusí znát heslo uživatele root
	\item použití sudo může být monitorováno
	\item je možné definovat, kteří uživatelé mohou provádět jaké úlohy
\end{itemize}

\subsection{Sudo a pluginy}
Sudo Plugin API\footnote{http://www.sudo.ws/sudo\_plugin.man.html}
umožňuje vytvoření vlastního modulu, který bude definovat vlastní správu
politik.  To jaké pluginy bude sudo používat je možné konfigurovat pomocí
souboru \textit{/etc/sudo.conf}.

\subsection{Distribuce sudo pravidel mezi více počítači}
Každý systém, na kterém chci využívat SUDO, musí mít soubor obsahující sudo
pravidla. Sudo ovšem nemá nativní způsob jak tyto pravidla šířit mezi více
systémy, alčkoliv existuje několik způsobů jako toho dosáhnout:

\subsubsection{Manuální distribuce}
Administrátor spravující více systémů může může sudo pravidla šířit mezi více systémy pomocí nástrojů jako je cron, scp, rsync, atd. Tento způsob je ovšem velmi pracný.

\subsubsection{Centrálním uložiště}
Je možné využít nástrojů, které automaticky sledují a distribuují soubory mezi
vice systémy. Mezi tyto nástroje patří např.
Puppet\footnote{http://puppetlabs.com/puppet/what-is-puppet}
 
\subsubsection{LDAP adresář}
Ve většině firemních prostředí se dnes používá adresářových služeb k
synchronizaci uživatelů, skupin a dalších sdílených informací. Sudo pravidla
mohou být na tomto serveru také uložena.  Přistoupit k nim je možné pomocí
protokolu LDAP\footnote{Lightweight Directory Access Protocol, which is an
application protocol for querying and modifying directory services.}. Při
použití adresářového serveru je možné sudo
pravidla centrálně spravovat a globálně k nim přistupovat. Sudo ovšem takto
nativně nepracuje proto je potřeba plugin. Sudo má plugin pro použítí s
openLDAP. Schéma v jakém jsou sudo pravidla na LDAP serveru uložena je pevně
specifikováno a přináší následující nevýhody: 

\begin{enumerate} 
	\item problém s	tím překrýváním lokálních uživatelů 
	\item další
\end{enumerate}

Pokud chce administrátor tento způsob správy politik využít, pak musí dané
schéma dodržet i se všemi jeho nevýhodami.

\subsection{Why is SUDO in LDAP better solution?}
LDAP is characterized as a "write-once-read-many-times" service. It's eminently
suitable for maintaining data which are not changed very often. Sudo rules are
set once by system's administrator but accessed by users many times.
Administrator will also want to edit the rules but not as often as client will
access it.

Directories are faster that databases because they don't require consistency as
much as relational or transactional databases. It doesn't use transactions,
locking or roll-backs\footnote{www.zytrax.com/books/ldap/ch2/}.


\subsection{What is meant by LDAP server?}
Technically, LDAP is just a protocol that defines the method by which directory
data is accessed. It also defines and describes how data is represented in the
directory service.


\section{FreeIPA}
FreeIPA používá vlastní schéma a proto musela napsat i vlastní plugin sss.

Výhody použití FreeIPA pro správu sudo pravidel - u sudo+openldap musím vytvořit
schéma a ručně jej přidat mezi ostatní schémata + přidat o tom záznam do
/etc/sldap.conf


\end{document}
