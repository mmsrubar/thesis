\documentclass[10pt,a4paper,notitlepage]{article}

\usepackage[left=1.5cm,text={18cm, 25cm},top=2.5cm]{geometry}
\usepackage{color}
\usepackage[utf8]{inputenc}

\author{Michal Šrubař\\xsruba03@stud.fit.vutbr.cz}
\date{}	
\title{Improved integration of SSSD and SUDO}

\begin{document}
\maketitle

\section*{Možná struktura textu}
\begin{itemize}
	\item	Co je smyslem této BP.
	\item Uživatelé, root, jak dát běžným uživatelům vyšší práva
	\item Sudo, su, alternativy
	\item Proč sudo?
	\item	Proč sudo v LDAPu? (alternativy jak a kam ukládát sudo pravidla, např.
		puppet)

	\begin{itemize}
		\item Složitost konfigurace a správy (všechno v textu, není GUI, které FreeIPA má)
		\item Firma s mnoha zaměstnanci chce centrálně spravovat informace a
			politiky o svých uživatelích (chtěla by mít vše na jednom místě a né
			informace v ldap, sudo pravidla v LDAPu ale nebylo by to nijak propojeno,
			konfiguráky v puppetu, atd...)
		\item sudo pravidla nejsou nijak propojena s ostatními informacemi, které
			mohou a často jsou v LDAPu uloženy (uživatelé, skupiny, atd)
	\end{itemize}
	
	\item Problémy a nedostatky defaultního LDAP shématu, které SUDO používá
	\begin{itemize}
		\item na uživatele se odkazuje pouze přes řetězec jména, není tam žádná
			provázanost s počítačem ze kterého dany uživatel je
		\item pokud LDAP spravuje i uživatele, a je tam stejný uživatel jako na
			lokální mašině, tak tam vzniká nějaký problém s překrytím těch uživatelů
		\item mezi uživateli a sudo pravidly neexistuje vazba, pokud odstraním
			uživatele, tak mně zůstane pravidlo s neexistujícím uživatelem
	\end{itemize}
	\item Proč sudo ve FreeIPA (?granularita?)
	\item Proč má IPA vlastní SUDO LDAP schema a jaké výhody přináší

	\begin{itemize}
		\item umožňuje např. povolit/zakázat pravidlo
	\end{itemize}
		
	\item Jak to funguje teď

	\begin{itemize}
		\item definice pravidla
		\item jak se uloží v LDAPu
		\item co tam bude jinak oproti tomu než bych to uložil do LDAPu s legacy
			schematem
		\item konverze do legacy schématu když k tomu přistoupím přes SSSD
		\item cachování
	\end{itemize}
	\item Proč musí IPA nové schéma přeložit do legacy schematu než jej pošle
		SSSD
	\item Jaké problémy to přináší
	\item Jaké úpravy je nutné provést
	\begin{itemize}
		\item u sudo bude třeba přenést něco z /lib do /usr (z ticketu: pe\_task,
			cron)
	\end{itemize}
	\item Forma reklamy na FreeIPA. Vyzdvihnout všechny přednosti
\end{itemize}

\section*{Hlavní problém, který moje BP řeší}
FreeIPA používá pro uložení SUDO pravidel v LDAPu svoje schéma, které přináší
mnoho vylepšení. \textbf{SSSD} ovšem toto schéma neumí zpracovat (FreeIPA
nejprve používala legacy schema pro uložení a pak navrhla nové). FreeIPA musí
nové schéma přeložit do legacy schématu, které sssd umí zpracovat. Převodem se
ovšem ztrácí mnoho informací, které potom způsobují problémy a brzí to tak
využítí celého potenciálu.

Jelikož SSSD neumí to FreeIPA schéma, tak se musí dodatečně stahovat mnoho
informací, což u rozsáhlé databáze způsobuje značné zpomalení. Úkolem není
navrhnout nové LDAP schéma pro SUDO, ale upravit SSSD daemon a SUDO tak, aby
podporovaly LDAP schéma, které používá IPA.
\\
\\
Hodně zjednodušeně bude výstupen filter pro \emph{ldapsearch}.


\section{Administration of Linux}
On most of today's linux and Unix-like operating
systems\footnote{http://en.wikipedia.org/wiki/Unix-like} administrative tasks
are performed by special user called a \textbf{root}. Also known as a \textbf{superuser}. The user has
access to all commands and files on the
system\footnote{http://www.linfo.org/root.html}. If an other user wants to
perform administrative task than there are a few ways to do it: 
\begin{itemize}
	\item The user has to know password of the root account.
	\item Use special program (su, sudo)  that can increase user's authority.
\end{itemize}


\subsection{su (Substitute User)}
This program allows to a user to temporarily become a particular user. Most
often the particular user is the root. This means that su can give any user, that
know root's password,  full control over an operating  system.



\subsection{sudo}
Sudo allows to a user to execute a single command as a particular user. Security
poilicy which determines user's privilegies to run sudo, i.e. \textbf{sudo
rules}, are stored in configuration file \emph{/etc/sudoers}. On every system
that I want to use sudo the file has to exists.



\subsection{su vs. sudo}
Jedná se dva naprosto odlišné přístupy k tomu jak získat oprávnění uživatele
root.

\begin{enumerate}
	\item sudo umožňuje limitovat přístup zatímco su může uživateli dát plnou
		kontrolu nad systémem
	\item uživatel nemusí znát heslo uživatele root
	\item použití sudo může být monitorováno
	\item je možné definovat, kteří uživatelé mohou provádět jaké úlohy
\end{enumerate}


\subsection{Distribuce sudo pravidel mezi více počítači}
There is not a native way to distribute sudo rules among multiple systems. Though there is a few ways to do it:

 Manuální šíření
Administrátor spravující více systémů může může sudo pravidla šířit mezi více systémy pomocí nástrojů jako je cron, scp, rsync, atd. Tento způsob je ovšem velmi pracný.

Centrálním uložiště
Je možné využít nástrojů, které automaticky sledují a distribuují konfigurační soubory mezi vice systémy. Mezi tyto nástroje patří např. Puppet\footnote{http://puppetlabs.com/puppet/what-is-puppet}
 

1.	LDAP
Ve většině firemních prostředí se dnes používá adresářových služeb k synchronizaci uživatelů, skupin a dalších sdílených informací. Sudo pravidla mohou být na tomto serveru také uložena. 


 Přistoupit k nim můžeme pomocí protokolu LDAP\footnote{Lightweight
	Directory Access Protocol, which is an application protocol for querying and
modifying directory services.}. Při použití adresářového serveru je možné sudo
pravidla centrálně spravovat a globálně k nim přistupovat. Sudo ovšem takto
nativně nepracuje proto je potřeba plugin. Sudo má plugin pro použítí s
openLDAP. Schéma v jakém jsou sudo pravidla na LDAP serveru uložena je pevně
specifikováno a přináší následující nevýhody: 
\begin{enumerate} 
	\item problém s	tím překrýváním lokálních uživatelů 
\end{enumerate}

Pokud chce administrátor tento způsob správy politik využít, pak musí dané
schéma dodržet i se všemi jeho nevýhodami.



\subsection{FreeIPA}
FreeIPA používá vlastní schéma a proto musela napsat i vlastní plugin sss.


Výhody použití FreeIPA pro správu sudo pravidel - u sudo+openldap musím vytvořit
schéma a ručně jej přidat mezi ostatní schémata + přidat o tom záznam do
/etc/sldap.conf


Sudo Plugin API\footnote{http://www.sudo.ws/sudo\_plugin.man.html}
umožňuje vytvoření vlastního modulu, který bude definovat vlastní správu
politik.  To jaké pluginy bude sudo používat je možné konfigurovat pomocí
souboru \textit{/etc/sudo.conf}.




LDAP enabled directories:
- active directory
- 389 directory service
 všechno programy (adresáře) ke kterým je možno přistoupit pomocí LDAP protokolu

What is meant by LDAP server?
Technically, LDAP is just a protocol that defines the method by which directory data is accessed. It also defines and describes how data is represented in the directory service.

Why SUDO in LDAP server?
LDAP is characterized as a "write-once-read-many-times" service. It's eminently suitable for maintaining data which are not changed very often. Sudo rules are set once by system's administrator but accessed by users many times. Administrator will also want to edit the rules but not as often as client will access it.

Directories are faster that databases because they don't require consistency as much as relational or transactional databases. It doesn't use transactions, locking or roll-backs\footnote{www.zytrax.com/books/ldap/ch2/}.

TODO:
•	simple asynchronous replication process
•	LDAP provides a remote and local data access method that is standardized. It is thus possible to replace the LDAP implementation completely without affecting the external interface to the data. RDBMS systems provide local access standards, such as SQL, but remote interfaces are always proprietary.
10	Because LDAP uses standardized data access methods Clients and Servers may be sourced (or developed) independently. By extension of this point LDAP may be used to abstract the view of data contained in transaction oriented databases, say for the purpose of running user queries, while allowing the user to transparently (to the LDAP queries) change the transactional database supplier.
20	LDAP provides a method whereby data may be moved (delegated) to multiple locations without affecting any external access to that data. By using referral methods LDAP data can be moved to alternate LDAP servers by changing operational parameters only.
30	LDAP systems can be operationally configured to replicate data to one or more applications without adding either code or changing the external access to that data.




	
\end{document}
