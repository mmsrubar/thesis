%%%%%%%%%%%%%%%%%%%%%%%%%%%%%%%%%%%%%%%%%%%%%%%%%%%%%%%%%%%%%%%%%%%%%%%%%%%%%%
%%%%%%%%%%%%%%%%%%%%%%%%%%%%%%%%%%%%%%%%%%%%%%%%%%%%%%%%%%%%%%%%%%%%%%%%%%%%%%
%%
%% Ukázkový příklad dokumentace úkolu do předmětů IZP a IUS, 2010
%%
%% Upravená původní dokumentace od Davida Martinka.
%%%%%%%%%%%%%%%%%%%%%%%%%%%%%%%%%%%%%%%%%%%%%%%%%%%%%%%%%%%%%%%%%%%%%%%%%%%%%%
%%%%%%%%%%%%%%%%%%%%%%%%%%%%%%%%%%%%%%%%%%%%%%%%%%%%%%%%%%%%%%%%%%%%%%%%%%%%%%
\documentclass[12pt,a4paper,titlepage,final]{article}

% cestina a fonty
\usepackage[czech]{babel}
\usepackage[utf8]{inputenc}
% balicky pro odkazy
\usepackage[bookmarksopen,colorlinks,plainpages=false,urlcolor=blue,unicode]{hyperref}
\usepackage{url}
% obrazky
\usepackage[dvipdf]{graphicx}
% velikost stranky
\usepackage[top=3.5cm, left=2.5cm, text={17cm, 24cm}, ignorefoot]{geometry}
\usepackage{listings}


\begin{document}

%%%%%%%%%%%%%%%%%%%%%%%%%%%%%%%%%%%%%%%%%%%%%%%%%%%%%%%%%%%%%%%%%%%%%%%%%%%%%%
% titulní strana
\author{Michal Šrubař\\xsruba03@stud.fit.vutbr.cz}
\date{}	
\title{Improved integration of SUDO and SSSD}
\maketitle
%\input{title.tex}

%%%%%%%%%%%%%%%%%%%%%%%%%%%%%%%%%%%%%%%%%%%%%%%%%%%%%%%%%%%%%%%%%%%%%%%%%%%%%%
% obsah
\pagestyle{plain}
\pagenumbering{roman}
\setcounter{page}{1}
\tableofcontents

%%%%%%%%%%%%%%%%%%%%%%%%%%%%%%%%%%%%%%%%%%%%%%%%%%%%%%%%%%%%%%%%%%%%%%%%%%%%%%
% textova zprava
\newpage
\pagestyle{plain}
\pagenumbering{arabic}
\setcounter{page}{1}

%%%%%%%%%%%%%%%%%%%%%%%%%%%%%%%%%%%%%%%%%%%%%%%%%%%%%%%%%%%%%%%%%%%%%%%%%%%%%%
\section*{Hlavní problém, který moje BP řeší}
FreeIPA používá pro uložení SUDO pravidel v LDAPu svoje schéma, které přináší
mnoho vylepšení. \textbf{SSSD} ovšem toto schéma neumí zpracovat (FreeIPA
nejprve používala legacy sudo schema pro uložení a pak se navrhlo nové schéma).
FreeIPA musí nové schéma přeložit do legacy schématu (provádí to compat plugin),
které sssd umí zpracovat. Převodem se ovšem ztrácí mnoho informací, které potom
způsobují problémy a brzí to tak využítí celého potenciálu.

Jelikož SSSD neumí to FreeIPA schéma, tak se musí dodatečně stahovat mnoho
informací, což u rozsáhlé databáze způsobuje značné zpomalení. Úkolem není
navrhnout nové LDAP schéma pro SUDO, ale upravit SSSD daemon a SUDO tak, aby
podporovaly LDAP schéma, které používá IPA.

Hodně zjednodušeně bude výsledkem správný dotaz na LDAP adresář na IPA serveru,
který pro daného klienta stáhne všechno potřebné.


% ==============================================================================
\section{Administrace linuxového systému}
Na většině dnešních linuxových a
Unix-like\footnote{http://en.wikipedia.org/wiki/Unix-like} operačních systémech
provádí správu systému speciální uživatel zvaný \textbf{root}. Někdy také
označován jako \textbf{superuser}. Tento uživatel má přístup ke všem souborům a
může spouštět všechny příkazy operačního
systému\footnote{http://www.linfo.org/root.html}.  Ostatní uživatele systému
toto právo nemají, až na tyto vyjímky: \begin{itemize}
	\item uživatel zná heslo uživatele root
	\item uživatel použije program, který mu dočasně poskytne práva uživatele root
\end{itemize}


\subsection{su (Substitute User)}
Tento program umožňuje libovolnému uživateli stát se dočasně jiným uživatelem.
Vě většině případů je jiný uživatel právě uživatel root. To znamená, že program
su může dát libovolnému uživateli, který zná heslo uživatele root, plnou
kontrolu nad operačním systémem. In a nutshell sudo provides a way to give
selective root access by user/machine/command.

% ==============================================================================

\section{SUDO (Super User DO)}

The sudo allows for a user to execute one command as a different user. Security
policies which define which users are allowed to execute what commands are
called \emph{sudo rules}. These rules are stored in a configuration file
\texttt{/etc/sudoers}. The file should be edited only with tool \emph{visudo}
which provides automatic syntax checking.

\subsection{SUDO vs. su}
These are two different concepts of giving root access to a user.

\begin{itemize}
	\item sudo allows limited access but su gives a user a full control over the
		system
	\item none of users need to know the root's password
	\item using sudo can be logged
	\item it's possible to define which users can execute which commands
\end{itemize}


\subsection{Distribution of the sudoers file among multiple systems}
SUDO doesn't have a native way to distribute \emph{sudoers} file among multiple
clients so the administrators in corporate environments face a problem. How to
distribute the sudo rules to all machines they administers? There are a few
solutions:

\begin{enumerate} 
	\item Administrator can manually distribute the \emph{sudoers} file among the
		systems he administers with standard UNIX tools such as \emph{cron, scp,
		rsync} etc.
	\item Use tools such as
		Puppet\footnote{http://puppetlabs.com/puppet/what-is-puppet} which
		automatically watch and distribute files among multiple systems.
	\item Use a centralized database. If he does so administrator just need to
		insert sudo rules into the database and all the clients can use it
\end{enumerate}

\subsection{LDAP as a centralized database}
LDAP \footnote{Lightweight Directory Access Protocol, which is an application
protocol for querying and modifying directory services.}is characterized as a
"write-once-read-many-times" service. It's eminently suitable for maintaining
data which are not changed very often. Sudo rules are set once by system's
administrator but accessed by users many times. The administrator will also want
to edit the rules but not as often as clients will access it.

Directories are faster than databases because they don't require consistency as
much as relational or transactional databases. It doesn't use transactions,
locking or roll-backs\footnote{www.zytrax.com/books/ldap/ch2/}.


\subsection{SUDO on LDAP server}
In most today's organization environments there is an LDAP server which can be
used to centrally manage user's information. This information can be accessed
by LDAP protocol. SUDO itself has built-in support for processing sudo rules
stored on LDAP server.


\subsection{The benefits of having sudoers on LDAP server}
\begin{enumerate} 
	\item If the sudo rules are centralized than the administrator doesn't have to
		worry about distributing them.
	\item Looking up sudo rules is faster because sudo no longer needs to read
		the entire \emph{sudoers} file. There are only a few LDAP queries per
		invocation.
	\item If there is a typing error in \emph{sudoers} than sudo won't start. With
		LDAP it's not possible to load data into the LDAP directory which does not
		conform the sudoers schema so the proper syntax is guaranteed. Although you
		can still make a mistake in user name, host name or command. There is no
		need to use visudo\footnote{utility which provides save editing of the
		\emph{sudoers} file}.
\end{enumerate}

\subsection{Difference between LDAP sudoers and non-LDAP sudoers}
\begin{itemize} 
	\item The biggest difference, according to the LDAP RFC, is that LDAP ordering
		of attributes is arbitrary which means that you cannot expect that
		attributes are returned in any specific order. 
		Let's suppose that we have these two rules in the \emph{sudoers} file:

		\begin{lstlisting}[basicstyle=\ttfamily\small,frame=lines,showtabs=false]
			adam ALL=/bin/cat /etc/shadow
			adam ALL=!/bin/cat /etc/shadow/
		\end{lstlisting}

		There's an order in reading \emph{sudoers} which means user \texttt{adam} will not be
		able to print content of the \emph{/etc/shadow} file. The same rule in LDAP
		would look like this:

		\lstinputlisting[language=C,basicstyle=\ttfamily\small,frame=lines,showtabs=false]{srcs/sudorule_order.ldif}
		Since there's an arbitrary ordering of attributes there's no guarantee of what
		sudo rules the client will receive as the last one. That means we are not 
		sure whether the user \texttt{adam} will be able to print \texttt{shadow}
		file or not. This is the reason why there is the \texttt{SudoOrder}
		attribute. {\color{blue}Take a closer look at this!}

	\item \texttt{User\_Aliases, RunAs\_Aliases and Cmnd\_Aliases} are not supported
	\item \texttt{User\_Aliases} can be replaced with groups and netgroups which can
		also be stored in LDAP
	\item \texttt{Cmnd\_Aliases} are not needed because it's possible to add more
		\texttt{sudoCommand} in one \texttt{sudoRole}
	\item \texttt{/etc/sudoers} file uses global default options but in LDAP it's
		possible to specify per-entry options.
\end{itemize}

Having sudoers in LDAP has also a few drawbacks. These are discussed in section
\ref{sec:sudo_ldap_drawbacks}.

\subsection{Drawbacks of having sudoers in LDAP}\label{sec:sudo_ldap_drawbacks}
\begin{itemize} 
	\item Administrator has to configure everything via text files. Sudo does not
		provide any GUI.
	\item Corporations don't want to maintain their user's information a security
		policies separately. They want to have those kind of information at one place
		so it can be link together.
	\item If there are user's information and sudo rules stored in LDAP than there
		is no relationship among them.
		rules.
	\item \texttt{User\_Aliases, RunAs\_Aliases and Cmnd\_Aliases} are not supported
	\item \texttt{User\_Aliases} can be replaced with groups and netgroups which can
		also be stored in LDAP
	\item \texttt{Cmnd\_Aliases} are not needed because it's possible to add more
		\texttt{sudoCommand} in one \texttt{sudoRole}
	\item \texttt{/etc/sudoers} file uses global default options but in LDAP it's
		possible to specify per-entry options.
\end{itemize}

\subsection{Sudo a pluginy}
Sudo Plugin API\footnote{http://www.sudo.ws/sudo\_plugin.man.html}
umožňuje vytvoření vlastního modulu, který bude definovat vlastní správu
politik.  To jaké pluginy bude sudo používat je možné konfigurovat pomocí
souboru \textit{/etc/sudo.conf}.

\subsection{What is meant by LDAP server?}
Technically, LDAP is just a protocol that defines the method by which directory
data is accessed. It also defines and describes how data is represented in the
directory service.

{\color{blue}
\subsection{ADD or NOT?}
\begin{enumerate} 
	\item example of sudoers configuration and use?
	\item Explain meaning of all \texttt{sudoRole} attributes and SUDOers container?
\end{enumerate} 
}


\section{FreeIPA}
FreeIPA uses LDAP directory to store information such as users information for
authentication, dns records etc.

\subsection{Why was FreeIPA created}

\subsection{What is it?}

\subsection{sudoers in FreeIPA (the new scheme)}

\subsection{Advantages the new schema brings}
\begin{itemize}
	\item it's possible to disable a rule without deleting it
	\item Identities, Hosts and Commands uses DNs instead of string attributes
\end{itemize}
T
FreeIPA používá vlastní schéma a proto musela napsat i vlastní plugin sss.

Výhody použití FreeIPA pro správu sudo pravidel - u sudo+openldap musím vytvořit
schéma a ručně jej přidat mezi ostatní schémata + přidat o tom záznam do
/etc/sldap.conf

\subsection{How are SUDO rules stored in LDAP on IPA server?}
IPA uses two containers for storing SUDO rules. A \emph{sudoers} container
which contains SUDO rules in legacy scheme which consists of these attributes:
\begin{itemize}
	\item sudoHost
	\item sudoCommand
	\item sudoUser
\end{itemize}
Then there is a \emph{sudo} container and this contains 

\subsection{Known problems}


\section{SSSD}
\subsection{What is it?}

\subsection{The benefits of the SSSD}
\textit{unified configuration - instead of configuring the SSSD to perform account
lookups and then configuring /etc/ldap.conf to perform sudo rules lookups, the
user only configures one client piece - the SSSD. The SSSD also provides several
advanced features that might not be available in other LDAP client packages,
such as the support for server discovery using DNS SRV requests or advanced
server fail over, which lets the admin define several servers that are tried in
descending order of preference and then stick to the working server, by Kuba}

\section{Jak to funguje teď}
\subsection{}
\subsection{definice pravidla}
\subsection{jak se uloží v LDAPu}
\subsection{co tam bude jinak oproti tomu než bych to uložil do LDAPu s legacy schematem}
\subsection{konverze do legacy schématu když k tomu přistoupím přes SSSD}
\subsection{cachování}
	
\subsection{compat plugin}
\textbf{Ten compat plugin pracuje za běhu IPA nebo ty pravidla přeloží až, když
o ně požádá SSSD?}

Myslim ze cely compat strom se generuje v zavislosti na dotazech klientu. Tj
na serveru je plugin, ktery vi, ze kdyz dojde dotaz na ou=sudoers,\$DC,
tak se ma interne zeptat do IPA sudo kontejneru a pokud neco dostane zpet,
vlozi vysledek do compat pluginu, kde si ho precte klient co provadel search.

Klient nemusi byt jen sudo, muze to byt cokoliv, co se umi bavit LDAP
protokolem.

\subsection{jak jsou uložena pravidla v IPA}
\textbf{V kontejneru sudores sjou povolené pravidla v takovém formátu, které umí
zpracovat SUDO. V sudo kontejneru jsou pravidla ve formátu, který používá IPA.
Ten plugin to překládá za běhu a SSSD si to tahá ze sudoers kontejneru? Když
potřebuje další informace, tak si je dodatečně stáhne?}

Jo, myslim ze jsi to popsal dobre.

\subsection{externí uživatele}
\textbf{Funguji momentálně ti externí uživatele. Pravidlo se mně
stáhne v pořádku, ale sudo mně řekne, že uživatel není v SUDOers.}

Externi uzivatele jsou hlavne na to, kdyz mas nektere uzivatele
ulozene na jinem serveru nez IPA (typicky Active Directory). Nevim
jestli jde pomoci externich uzivatelu vlozit treba uzivatele z
/etc/passwd, ale spis bych rekl, ze ne.

Myslim ze tomuhle se v praci nemusis nijak venovat, max popsat, ze to tam je.



\section{Proposed changes}
\subsection{u sudo bude třeba přenést něco z /lib do /usr (z ticketu: pe\_task,
									cron)}
	


\section*{Možná struktura textu}
\begin{itemize}
	\item Problémy a nedostatky defaultního LDAP shématu, které SUDO používá
	\begin{itemize}
					\item na uživatele se odkazuje pouze přes řetězec jména, není tam žádná
									provázanost s počítačem ze kterého dany uživatel je
					\item pokud LDAP spravuje i uživatele, a je tam stejný uživatel jako na
									lokální mašině, tak tam vzniká nějaký problém s překrytím těch uživatelů
					\item mezi uživateli a sudo pravidly neexistuje vazba, pokud odstraním
									uživatele, tak mně zůstane pravidlo s neexistujícím uživatelem
	\end{itemize}
\end{itemize} 

\end{document}
