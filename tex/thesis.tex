%%%%%%%%%%%%%%%%%%%%%%%%%%%%%%%%%%%%%%%%%%%%%%%%%%%%%%%%%%%%%%%%%%%%%%%%%%%%%%
%%%%%%%%%%%%%%%%%%%%%%%%%%%%%%%%%%%%%%%%%%%%%%%%%%%%%%%%%%%%%%%%%%%%%%%%%%%%%%
%%
%% Ukázkový příklad dokumentace úkolu do předmětů IZP a IUS, 2010
%%
%% Upravená původní dokumentace od Davida Martinka.
%%%%%%%%%%%%%%%%%%%%%%%%%%%%%%%%%%%%%%%%%%%%%%%%%%%%%%%%%%%%%%%%%%%%%%%%%%%%%%
%%%%%%%%%%%%%%%%%%%%%%%%%%%%%%%%%%%%%%%%%%%%%%%%%%%%%%%%%%%%%%%%%%%%%%%%%%%%%%
\documentclass[12pt,a4paper,titlepage,final]{article}

% cestina a fonty
\usepackage[czech]{babel}
\usepackage[utf8]{inputenc}
% balicky pro odkazy
\usepackage[bookmarksopen,colorlinks,plainpages=false,urlcolor=blue,unicode]{hyperref}
\usepackage{url}
% obrazky
\usepackage[dvipdf]{graphicx}
% velikost stranky
\usepackage[top=3.5cm, left=2.5cm, text={17cm, 24cm}, ignorefoot]{geometry}


\begin{document}

%%%%%%%%%%%%%%%%%%%%%%%%%%%%%%%%%%%%%%%%%%%%%%%%%%%%%%%%%%%%%%%%%%%%%%%%%%%%%%
% titulní strana
\author{Michal Šrubař\\xsruba03@stud.fit.vutbr.cz}
\date{}	
\title{Zdokonalení integrace SSSD a SUDO}
\maketitle
%\input{title.tex}

%%%%%%%%%%%%%%%%%%%%%%%%%%%%%%%%%%%%%%%%%%%%%%%%%%%%%%%%%%%%%%%%%%%%%%%%%%%%%%
% obsah
\pagestyle{plain}
\pagenumbering{roman}
\setcounter{page}{1}
\tableofcontents

%%%%%%%%%%%%%%%%%%%%%%%%%%%%%%%%%%%%%%%%%%%%%%%%%%%%%%%%%%%%%%%%%%%%%%%%%%%%%%
% textova zprava
\newpage
\pagestyle{plain}
\pagenumbering{arabic}
\setcounter{page}{1}

%%%%%%%%%%%%%%%%%%%%%%%%%%%%%%%%%%%%%%%%%%%%%%%%%%%%%%%%%%%%%%%%%%%%%%%%%%%%%%
\section*{Hlavní problém, který moje BP řeší}
FreeIPA používá pro uložení SUDO pravidel v LDAPu svoje schéma, které přináší
mnoho vylepšení. \textbf{SSSD} ovšem toto schéma neumí zpracovat (FreeIPA
nejprve používala legacy sudo schema pro uložení a pak se navrhlo nové schéma).
FreeIPA musí nové schéma přeložit do legacy schématu (provádí to compat plugin),
které sssd umí zpracovat. Převodem se ovšem ztrácí mnoho informací, které potom
způsobují problémy a brzí to tak využítí celého potenciálu.

Jelikož SSSD neumí to FreeIPA schéma, tak se musí dodatečně stahovat mnoho
informací, což u rozsáhlé databáze způsobuje značné zpomalení. Úkolem není
navrhnout nové LDAP schéma pro SUDO, ale upravit SSSD daemon a SUDO tak, aby
podporovaly LDAP schéma, které používá IPA.
\\
\\
Hodně zjednodušeně bude výsledkem správný dotaz na LDAP adresář na IPA serveru,
který pro daného klienta stáhne všechno potřebné.


% ==============================================================================
\section{Administrace linuxového systému}
Na většině dnešních linuxových a
Unix-like\footnote{http://en.wikipedia.org/wiki/Unix-like} operačních systémech
provádí správu systému speciální uživatel zvaný \textbf{root}. Někdy také
označován jako \textbf{superuser}. Tento uživatel má přístup ke všem souborům a
může spouštět všechny příkazy operačního
systému\footnote{http://www.linfo.org/root.html}.  Ostatní uživatele systému
toto právo nemají, až na tyto vyjímky: \begin{itemize}
	\item uživatel zná heslo uživatele root
	\item uživatel použije program, který mu dočasně poskytne práva uživatele root
\end{itemize}


\subsection{su (Substitute User)}
Tento program umožňuje libovolnému uživateli stát se dočasně jiným uživatelem.
Vě většině případů je jiný uživatel právě uživatel root. To znamená, že program
su může dát libovolnému uživateli, který zná heslo uživatele root, plnou
kontrolu nad operačním systémem. In a nutshell sudo provides a way to give
selective root access by user/machine/command.


\subsection{sudo (Super User DO)}
Sudo umožňuje uživateli provést jeden příkaz jako jiný uživatel. Bezpečností
politiky, které definují kteří uživatelé mohou spouštět jaké příkazy se nazývají 
\textbf{sudo pravidla}. Tyto pravidla jsou uloženy v souboru
\emph{/etc/sudoers}. Sudo pravidla by měla být upravována výhradně pomocí
utility \textbf{visudo}, která zároveň provádí kontrolu syntaxe tohoto souboru.

\subsubsection{výhody sudo oproti su}
Jedná se dva naprosto odlišné přístupy k tomu jak získat oprávnění uživatele
root.

\begin{itemize}
	\item sudo umožňuje limitovat přístup zatímco su může uživateli dát plnou
		kontrolu nad systémem
	\item none of users need to know the root's password
	\item using sudo can be logged
	\item it's possible to define which users can execute which commands
\end{itemize}

% ==============================================================================

\section{SUDO}

\subsection{Distribute sudoers file among multiple systems}
SUDO doesn't have native way to distribute \emph{sudoers} file among multiple
clients so the administrators in corporate environemnts faces a problem. How to
distribute the sudo rules to all machines they administers? There is a few
solutions:

\begin{enumerate} 
	\item Administrator can manualy distribute the \emph{sudoers} file among the
		systems he administers with starndar UNIX tools such as \emph{cron, scp,
		rsync} etc.
	\item Use tools such as
		Puppet\footnote{http://puppetlabs.com/puppet/what-is-puppet} which
		automaticaly watch and distribute files among multiple systems.
	\item store sudo rules in centralized database such as
		LDAP\footnote{Lightweight Directory Access Protocol, which is an application
		protocol for querying and modifying directory services.}.
\end{enumerate}

\subsection{Why is SUDO in LDAP better solution?}
LDAP is characterized as a "write-once-read-many-times" service. It's eminently
suitable for maintaining data which are not changed very often. Sudo rules are
set once by administrator of the system but accessed by users many times.
Administrator will also want to edit the rules but not as often as clients will
access it.

Directories are faster that databases because they don't require consistency as
much as relational or transactional databases. It doesn't use transactions,
locking or roll-backs\footnote{www.zytrax.com/books/ldap/ch2/}.


\subsection{SUDO on LDAP server}
In most today's organization environments there is a LDAP server which can be
used to centraly manage user's information. These information can be accessed
with LDAP protocol.
\\\\
{\color{blue}{Explain meaning of all sudo LDAP attributes?}}

\subsection{Benefits of having sudoers on LDAP server}
\begin{enumerate} 
	\item Looking up sudo rules is faster because sudo no longer needs to read
		entire \emph{sudoers} file. There are only a few LDAP queries per
		invocation.
	\item If there is a typing error in \emph{sudoers} than sudo won't start. With
		LDAP it's not possible to load data into the LDAP directory which does not
		conform the sudoers schema so the proper syntax is guaranted. Although you
		can still make a mistake in user name, host name or command. There is no
		need to use visudo\footnote{utility which provides save editing of the
		\emph{sudoers} file}.
\end{enumerate}

\subsection{Difference between sudoers and LDAP}
\begin{itemize} 
	\item The biggest difference, according to the LDAP RFC, is that LDAP ordering
		of attributes is arbitrary which means that you can not expect that
		attributes are returned in any specific order. 
		Let's suppose that we have these two rules in \emph{sudoers} file:
		\\\\\quad
		\texttt{adam ALL=/bin/cat /etc/shadow}\\
		\texttt{adam ALL=!/bin/cat /etc/shadow}\\
		\\
		There's an order in reading \emph{sudoers} which means adam will not be able
		to print content of the \emph{/etc/shadow} file.
	\item \texttt{User\_Aliases, RunAs\_Aliases and Cmnd\_Aliases} are not supported
	\item \texttt{User\_Aliases} can be replaced with groups and netgroups which can
		also be stored in LDAP
	\item \texttt{Cmnd\_Aliases} are not needed becase it's possible to add more
		\texttt{sudoCommand} in one \texttt{sudoRole}
	\item \texttt{/etc/sudoers} file uses global default options but in LDAP it's
		possible to specify per-entry options.
\end{itemize}



\subsubsection{Centrálním uložiště}

\subsubsection{LDAP adresář}
Ve většině firemních prostředí se dnes používá adresářových služeb k
synchronizaci uživatelů, skupin a dalších sdílených informací. Sudo pravidla
mohou být na tomto serveru také uložena.  Přistoupit k nim je možné pomocí
protokolu LDAP. Při
použití adresářového serveru je možné sudo
pravidla centrálně spravovat a globálně k nim přistupovat. Sudo ovšem takto
nativně nepracuje proto je potřeba plugin. Sudo má plugin pro použítí s
openLDAP. Schéma v jakém jsou sudo pravidla na LDAP serveru uložena je pevně
specifikováno a přináší následující nevýhody: 

\begin{enumerate} 
	\item problém s	tím překrýváním lokálních uživatelů 
	\item další
\end{enumerate}

Pokud chce administrátor tento způsob správy politik využít, pak musí dané
schéma dodržet i se všemi jeho nevýhodami.

\subsection{Sudo a pluginy}
Sudo Plugin API\footnote{http://www.sudo.ws/sudo\_plugin.man.html}
umožňuje vytvoření vlastního modulu, který bude definovat vlastní správu
politik.  To jaké pluginy bude sudo používat je možné konfigurovat pomocí
souboru \textit{/etc/sudo.conf}.


\subsection{What is meant by LDAP server?}
Technically, LDAP is just a protocol that defines the method by which directory
data is accessed. It also defines and describes how data is represented in the
directory service.


\section{FreeIPA}
FreeIPA používá vlastní schéma a proto musela napsat i vlastní plugin sss.

Výhody použití FreeIPA pro správu sudo pravidel - u sudo+openldap musím vytvořit
schéma a ručně jej přidat mezi ostatní schémata + přidat o tom záznam do
/etc/sldap.conf

\subsection{How are SUDO rules stored in LDAP on IPA server?}
{\Large{section keywords:} LDAP, IPA, SUDO legacy schema, IPA SUDO schema,
container}
\\
IPA uses two containers for storing SUDO rules. A \emph{sudoers} container
which contains SUDO rules in legacy scheme which consists of these attributes:
\begin{itemize}
	\item sudoHost
	\item sudoCommand
	\item sudoUser
\end{itemize}
Then there is a \emph{sudo} container and this contains 

\section{Benefits of the SSSD}
\textit{unified configuration - instead of configuring the SSSD to perform account
lookups and then configuring /etc/ldap.conf to perform sudo rules lookups, the
user only configures one client piece - the SSSD. The SSSD also provides several
advanced features that might not be available in other LDAP client packages,
such as the support for server discovery using DNS SRV requests or advanced
server fail over, which lets the admin define several servers that are tried in
descending order of preference and then stick to the working server, by Kuba}

\section{Changes}
{\Large{section keywords:}} changes, proposal, sssd, sudo, ipa, ldap sudo scheme,
sudoers container, sudo container

\section*{Možná struktura textu}
\begin{itemize}
        \item        Smysl BP.
        \item Uživatelé, root, jak dát běžným uživatelům vyšší práva
        \item Sudo, su, alternativy
        \item Proč sudo?
        \item        Proč sudo v LDAPu? (alternativy jak a kam ukládát sudo pravidla, např.
                puppet)

        \begin{itemize}
                \item Složitost konfigurace a správy (všechno v textu, není GUI, které FreeIPA má)
                \item Firma s mnoha zaměstnanci chce centrálně spravovat informace a
                        politiky o svých uživatelích (chtěla by mít vše na jednom místě a né
                        informace v ldap, sudo pravidla v LDAPu ale nebylo by to nijak propojeno,
                        konfiguráky v puppetu, atd...)
                \item sudo pravidla nejsou nijak propojena s ostatními informacemi, které
                        mohou a často jsou v LDAPu uloženy (uživatelé, skupiny, atd)
        \end{itemize}
        
        \item Problémy a nedostatky defaultního LDAP shématu, které SUDO používá
        \begin{itemize}
                \item na uživatele se odkazuje pouze přes řetězec jména, není tam žádná
                        provázanost s počítačem ze kterého dany uživatel je
                \item pokud LDAP spravuje i uživatele, a je tam stejný uživatel jako na
                        lokální mašině, tak tam vzniká nějaký problém s překrytím těch uživatelů
                \item mezi uživateli a sudo pravidly neexistuje vazba, pokud odstraním
                        uživatele, tak mně zůstane pravidlo s neexistujícím uživatelem
        \end{itemize}
        \item Projekt FreeIPA
        \item Proč sudo ve FreeIPA (?granularita?)
        \item Proč má IPA vlastní SUDO LDAP schema a jaké výhody přináší

        \begin{itemize}
                \item umožňuje např. povolit/zakázat pravidlo
        \end{itemize}
                
        \item Jak to funguje teď

        \begin{itemize}
                \item definice pravidla
                \item jak se uloží v LDAPu
                \item co tam bude jinak oproti tomu než bych to uložil do LDAPu s legacy
                        schematem
                \item konverze do legacy schématu když k tomu přistoupím přes SSSD
                \item cachování
        \end{itemize}
        \item Proč musí IPA nové schéma přeložit do legacy schematu než jej pošle
                SSSD
        \item Jaké problémy to přináší
        \item Jaké úpravy je nutné provést
        \begin{itemize}
                \item u sudo bude třeba přenést něco z /lib do /usr (z ticketu: pe\_task,
                        cron)
        \end{itemize}
        \item Vyzdvihnout všechny přednosti
\end{itemize} 

\end{document}
