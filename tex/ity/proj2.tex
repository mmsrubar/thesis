\documentclass[11pt,a4paper,twocolumn]{article}

\usepackage[left=1.5cm,text={18cm, 25cm},top=2.5cm]{geometry}
\usepackage{color}
\usepackage[czech]{babel}
\usepackage[IL2]{fontenc}
\usepackage[utf8]{inputenc}
\usepackage{amsmath}
\usepackage{amsthm}
\usepackage{amsfonts}
\usepackage{mdwlist}

\newcommand{\myUrl}[1]{{\color{blue}{#1}}}
\newcommand{\myuv}[1]{\quotedblbase #1\textquotedblleft}

\newtheorem{myproof}{Důkaz}
\newtheorem{mydef}{Definice}[section]
\newtheorem{myalg}[mydef]{Algoritmus}
\newtheorem{mylemma}{Věta}

\hyphenation{Prvek}
%\email{krena@fit.vutbr.cz}

\begin{document}
\begin{titlepage}
\begin{center}
\Large\textsc{Vysoké učení technické v~Brně\\
Fakulta informačních technologií}\\
\vspace{\stretch{0.382}}
%\vspace{\stretch{0.371}}
\LARGE Typografie a publikování \,--\, 2. projekt\\[3mm]
{\Huge Sazba dokumentů s~matematickými výrazy}\\
%\vspace{\stretch{0.629}}
\vspace{\stretch{0.618}}
%{\Large \time}
%\time
\end{center}
\Large\today\hfill Michal Šrubař
\bigskip
\end{titlepage}

\section{Úvod}
Tato úloha je zaměřena na sazbu titulní strany a~textů, které obsahují 
matematické vzorce, rovnice (jako třeba (\ref{eq:1}), (\ref{eq:2}) a
(\ref{eq:3})) a prostředí (například definice \ref{sec:3:def:1} na straně 
\pageref{sec:3:def:1} v~sekci \ref{sec:3}).

Na titulní straně je využito sázení nadpisu podle optického středu 
s~využitím \emph{zlatého řezu}. Tento postup byl probírán na přednášce. Pro 
sazbu matematických elementů byly využity balíky \AmS-\LaTeX u. 

\section{Plynulý matematický text}
Zásady pro sazbu matematiky v~plynulém textu odpovídají zásadám pro 
smíšenou sazbu. V~\LaTeX u si můžeme sazbu opakovaných symbolů a jejich 
posloupností zjednodušit zavedením vlastních příkazů.

Pro množinu $M$ označuje card($M$) kardinalitu $M.$ Pro množinu $M$ 
reprezentuje $M^*$ volný monoid generovaný množinou $M$ s~operací 
konkatenace. Prvek identity ve volném monoidu $M^*$ značíme symbolem
$\varepsilon$. Nechť $M^+=M^*-\{\varepsilon\}$. Algebraicky je tedy $M^+$ 
volná pologrupa generovaná množinou $M$ s~operací konkatenace.
Konečnou neprázdnou množinu $M$ nazvěme \emph{abeceda}.
Pro $w\in M^*$ označuje $|w|$ délku řetězce $w$. Pro $W\subseteq M$ 
označuje occur($w, W$) počet výskytů symbolů z~$W$ v~řetězci $w$ a 
$sym(w,i)$ určuje $i$-tý symbol řetězce $w$; například $sym(abcd, 3) = c$.

\section{Sazba definic a vět}\label{sec:3}
Pro sazbu definic a vět slouží balík \texttt{amsthm}.
\begin{mydef}\label{sec:3:def:1}
\emph{Bezkontextová gramatika} je čtveřice $G=(V,T,P,S)$, kde
\begin{description*} \itemsep0pt \parskip0pt \parsep0pt
\item[$V$] je totální abeceda,
\item[$T\subseteq V$] je abeceda terminálů,
\item[$S\in(V-T)$] je startující symbol,
\item[$P$] je konečná množina pravidel tvaru $q\colon A\rightarrow\alpha$,~kde
$A\in(V-T)$, $\alpha\in V^*$a $q$ je návěští tohoto pra\-vidla.
\end{description*}

\noindent Nechť $N=V-T$ značí abecedu neterminálů. Pokud
$q:A\rightarrow\alpha\in P$, $\gamma$, $\delta\in V^*$, $G$ provádí 
derivační krok z~ $\gamma A\delta$ do $\gamma\alpha\delta$ podle pravidla 
$q:A\rightarrow\alpha$, symbolicky píšeme $\gamma A\delta  \Rightarrow
\gamma\alpha\delta$
$[q:A\rightarrow\alpha]$ nebo zjednodušeně $\gamma A\delta \Rightarrow \gamma\alpha\delta$.
Standardním způsobem definujeme $\Rightarrow^n$, kde $n\geq0$. Dále definujeme 
tranzitivní uzávěr $\Rightarrow^+$ a~tranzitivně-reflexivní uzávěr $\Rightarrow^*$.
\end{mydef}

Algoritmus můžeme uvádět textově, podobně jako definice, nebo lze použít
pseudokódu vysázeného ve vhodném prostředí (například \texttt{algorithm2e}).

\begin{myalg}\emph{Ověření bezkontextovosti gramatiky.} Mějme gramatiku $G=(N,T,P,S)$.
\begin{enumerate}
	\item\label{sec:3,alg:1} Pro každé pravidlo $p\in P$ proveď test, zda $p$ na levé straně
		obsahuje právě jeden symbol z~$N$.
	\item Pokud všechna pravidla splňují podmínku z~kro\-ku \ref{sec:3,alg:1}, 
		tak je gramatika $G$ bezkontextová.
\end{enumerate}
\end{myalg}

\begin{mydef}
\emph{Jazyk} definovaný gramatikou $G$ definujeme jako 
$L(G) = \{w\in T^*\;|\;S\Rightarrow^*w\}$.
\end{mydef}


\subsection{Podsekce obsahující větu}
Věty a definice mohou mít vzájemně nezávislé číslování. Důkaz se obvykle uvádí hned za větou.

\begin{mydef}
Nechť $L$ je libovolný jazyk. $L$ je bezkontextový jazyk, když a
jen když $L=L(G)$, kde $G$ je libovolná bezkontextová gramatika.
\end{mydef}

\begin{mydef}
Množinu $\mathcal{L}_{CF} = \{L|L$ je bezkon\-textový jazyk\} nazýváme \emph{třídou 
bezkontextových jazyků.}
\end{mydef}

\begin{mylemma}\label{pumping_lemma}
Nechť $L_{abc} = \{a^nb^nc^n\ | n \geq 0\}$. Platí, že $L_{abc} \in \mathcal{L}_{CF}$.
\end{mylemma}

\begin{proof}
Důkaz se provede pomocí Pumping lemma pro bezkontextové jazyky a je zřejmý, což
implikuje pravdivost věty \ref{pumping_lemma}.
\end{proof}


\section{Rovnice a odkazy}
Složitější matematické formulace sázíme mimo plynulý text. Lze umístit několik 
výrazů na jeden řádek, ale pak je třeba tyto vhodně oddělit, například příkazem
\verb|\quad|.
$$\sqrt[3]{_{4}^{1}b_{2}^{3}} \quad \mathbb N=\{1,2,3,\dots\!\} \quad 
x^{y^{y}}\neq x^{yy} \quad z_{i_{j}}\not\equiv  z_{ij}$$

V~rovnici (\ref{eq:1}) jsou využity tři typy závorek s~různou explicitně definovanou velikostí.

\begin{eqnarray}\label{eq:1}
s(x) & = & \sqrt{\frac{1}{k}\sum_{i=1}^{k}p_{i}(x_{i}-x)^2}\nonumber\\
x & = & -\bigg\{\Big[\big(a*b\big)^c-d\Big]+1\bigg\}
\end{eqnarray}

V~této větě vidíme, jak vypadá implicitní vysázení limity 
$\lim_{n\rightarrow\infty}f(n)$ v~normálním odstavci textu. Podobně je to i 
s~dalšími symboly jako $\sum_{1}^{n}$ či $\bigcup _{A  \in \mathcal{B}}$.
 \newpage %FIXME: remove it

\noindent V~případě vzorce $\lim_{x\rightarrow 0}\limits\frac{\sin x}{x}=1$ 
jsme si vynutili méně úspornou sazbu příkazem \verb|\limits|.

\begin{eqnarray}
\int_{a}^{b}f(x)\,\mathrm{d}x &=& - \int\limits _b^a f(x)\,\mathrm{d}x\label{eq:2}\\
\overline{\overline{A}\wedge\overline{B}} &=& \overline{\overline{A}\vee\overline{B}}\label{eq:3}
\end{eqnarray}

Odkazy na číslované rovnice nebo matematické výrazy se mohou v~textu 
vyskytovat jak před, tak i~za jejich výskytem. Protože se rovnice číslují 
pomocí čísel v~kulatých závorkách, měly by mít tuto podobu i~odkazy na ně.

\section{Složené zlomky}
Při sázení složených zlomků dochází ke zmenšování použitého písma v~čitateli 
a jmenovateli. Toto chování není vždy žádoucí, protože některé zlomky potom 
mohou být obtížně čitelné.

V~těchto případech je možné nastavit standardní stupeň písma v~podvýrazech 
ručně pomocí příkazu \verb|\displaystyle| u~vysázených vzorců nebo pomocí 
\verb|\textstyle| u~vzorců, které jsou součástí textu. Srovnejte:
\begin{center}$
\displaystyle\frac{\frac{x+y}{(a+b+c)^3}-\frac{x-y}{\frac{ac}{b}}}{1-\frac{a+b}{c(a-b)}}\quad
\displaystyle \frac{\displaystyle\frac{x+y}{(a+b+c)^3}-\displaystyle\frac{x-y}{\displaystyle\frac{ac}{b}}}{1-\displaystyle\frac{a+b}{c(a-b)}}
$\end{center}
Tento postup lze použít nejen u~zlomků.

$$\prod_{i=0}^{m-1}(n-i)=\underbrace{n(n-1)(n-2)\,\dots(n-m+1)}_{\displaystyle\text{$m$ je počet
činitelů}}$$

\section{Matice}
Pro sázení matic se velmi často používá prostředí \texttt{array} a závorky 
(\verb|\left|, \verb|\right|). Tyto příkazy vždy tvoří pár a nelze je použít 
samostatně.

$$\begin{pmatrix}
\widetilde{c+d} & a-b\\ 
 \aleph & \tilde{b}\\ 
 \vec{a}& \frac{a}{b} \\
\vartheta &  \underrightarrow{AC} \\
\end{pmatrix}$$

$$\textbf{\text{C}}=\begin{Vmatrix}
c_{11} & c_{12} & \dots & c_{1n}\\ 
c_{21} & c_{22} & \dots & c_{2n}\\ 
\vdots  & \vdots & \ddots & \vdots\\ 
c_{m1} & c_{m2} &  \dots&c_{mn}
\end{Vmatrix}$$

$$\begin{vmatrix}
\ d &e\ \\ 
 t&u
\end{vmatrix} = du-et$$

Prostředí \texttt{array} lze úspěšně využít i jinde.

$$\binom{n}{k} = \textstyle\left\{ 
\begin{array}{l l}
	\displaystyle\frac{n!}{k!(n-k)!} & \quad \mbox{pro $ 0 \leq k~\leq n$} \\ 
	0 & \quad \mbox{pro $k < 0$ nebo $k > n$}\\
\end{array} \right. $$
 
\section{Závěrem}
V~případě, že budete potřebovat vyjádřit matema\-tickou konstrukci nebo symbol a 
nebude se Vám dařit jej nalézt v~samotném \LaTeX u, doporučuji prostudovat 
možnosti balíku maker \AmS-\LaTeX. Analogická poučka platí obecně pro 
jakoukoli matematickou konstrukci v~\TeX u.

\end{document}
